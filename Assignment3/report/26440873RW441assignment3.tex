\documentclass[conference]{IEEEtran}
\usepackage{cite}
\usepackage{amsmath,amssymb,amsfonts}
\usepackage{algorithmic}
\usepackage{graphicx}
\usepackage{textcomp}
\usepackage{xcolor}
\usepackage{float}
\usepackage{booktabs}
\usepackage{url}

\def\BibTeX{{\rm B\kern-.05em{\sc i\kern-.025em b}\kern-.08em
    T\kern-.1667em\lower.7ex\hbox{E}\kern-.125emX}}

\begin{document}

\title{ML441 Assignment 3}

\author{\IEEEauthorblockN{RH Buhr, 26440873}
\IEEEauthorblockA{\textit{BDatSci Programme, 4th Year} \\
\textit{Stellenbosch University}\\
Stellenbosch, South Africa \\
26440873@sun.ac.za}
}

\maketitle

\begin{abstract}

\end{abstract}

\section{\textbf{Introduction}}

\section{\textbf{Background}}

\subsection{\textbf{Datasets}}

The recurrent neural networks were trained and evaluated across five distinct datasets to provide a comprehensive assessment of their performance across diverse tasks.

\subsubsection{\textbf{AMD Stock Data}}

The AMD dataset records AMD (Advanced Micro Devices) stock data over a period of 40 years. The dataset records 10098 instances of 7 variables:

\begin{itemize}
    \item \texttt{Date}: Date the instance was recorded in the format 'yyyy-mm-dd'.
    \item \texttt{Open}: The price at which the stock first traded on the trading day.
    \item \texttt{High}: The highest price the stock reached during the trading day.
    \item \texttt{Low}: The lowest price the stock reached during the trading day.
    \item \texttt{Close}: The final price of the stock when the market closed on that day.
    \item \texttt{Adj Close}: The closing price adjusted for corporate actions like stock splits or dividends.
    \item \texttt{Volume}: The total number of shares traded on that day.
\end{itemize}

This is a regression problem, with \texttt{Adj Close} as the response, instead of \texttt{Close}, because it reflects the true economic value of the stock by incorporating adjustments, these adjustments ensure the data more accurately reflects the stock market trends.

\begin{figure}[H]
    \centering
    \includegraphics[width=0.5\textwidth]{images/amd_adj_close.pdf}
    \caption{AMD Adjusted Closing Prices Over Time}
    \label{fig:amd_adj_close}
\end{figure}

Figure~\ref{fig:amd_adj_close} displays how AMD's adjusted closing price fluctuated from 1980 through 2020. The series is clearly non-stationary, with periods of sharp growth and declines. The overall trend is upward, even though there are multiple crashes. Most notably around the early 2000s dot-com bubble and the 2008 financial crisis. The defining upward trend, showing rapid growth, started in 2020.

To formally asses stationarity, and Augmented Dickey-Fuller (ADF) test was conducted. The ADF test is a widely used statistical procedure for detecting unit roots in time series data. If a series has a unit root, it means it behaves like a random walk and is thus non-stationary. The ADF statistic of the AMD stock dataset yielded a p-value of 0.05468, meaning the null hypothesis is not rejected at the 5\% significance level and the series is non-stationary.

The series also exhibits no clear periodic seasonal structure, which is visually evident. Overall the dataset demonstrates the characteristics of a typical non-stationary financial time series, with long-term upward trend, \cite{stock_market_dataset_amd}.

\subsubsection{\textbf{Air Quality}}

The Air Quality dataset records hourly averaged responses of a gas multi-sensor device deployed in an Italian city between March 2004 to February 2005. The dataset records 9358 instances with multiple variables including pollutants ($CO$, $NMHC$, $C_6H_6$, $NO_x$, $N_2$), sensor responses from five metal oxide sensors,as well as temperature, relative humidity, and absolute humidity.

For this study the focus is on \texttt{CO(GT)}, concentration of carbon monoxide in $mg/m^3$, making this a regression problem.

\begin{figure}[H]
\centering
\includegraphics[width=0.5\textwidth]{images/air_q_over_time.pdf}
\caption{Air Quality Over Time: CO(GT) with 24h and 7d rolling means}
\label{fig:air_q_over_time}
\end{figure}

Figure~\ref{fig:air_q_over_time} shows the hourly CO(GT) values along with 24-hour and 7-day rolling averages. The raw series is highly volatile, with frequent spikes while the rolling means reveal drifts in the $CO$ concentration. These shifts indicate that although the hourly data are noisy, there are underlying patterns driven by human activity and environmental causes.

\begin{figure}[H]
\centering
\includegraphics[width=0.5\textwidth]{images/air_q_seasonal.pdf}
\caption{Seasonal decomposition of CO(GT) series}
\label{fig:air_q_seasonal}
\end{figure}

Figure~\ref{fig:air_q_seasonal} shows the seasonal decomposition of the \texttt{CO(GT)} series. The trend component highlights longer-term fluctuations, where the average $CO$ levels decrease during mid 2004 and increase again toward late 2004 and early 2005. The seasonal component reveals clear short-term periods, consistent with daily and weekly cycles, possibly linked to human activity.

To assess stationarity an ADF test was done. The test yielded a p-value of $2.498 \times 10^{-16}$, meaning the null hypothesis of the unit root is rejected and the \texttt{CO(GT)} series can be considered stationary. Although the data exhibits short term cycles and noisy behavior, the statistical properties remain stable over time.

In summary the Air Quality dataset exhibits short-term variability, seasonal cycles and a stationary structure. This makes it a suitable candidate for time series modeling with recurrent neural networks.

\section{\textbf{Implementation}}

\section{\textbf{Empirical process}}

\section{\textbf{Results \& discussion}}

\section{\textbf{Conclusion}}

\bibliographystyle{plain}
\bibliography{references}
\vspace{12pt}

\end{document}
